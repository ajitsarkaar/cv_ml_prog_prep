\refstepcounter{chapter}
\chapter{Machine Learning}

\section{Resources}
\begin{itemize}
\item https://github.com/afshinea/stanford-cs-229-machine-learning
\item https://ml2.inf.ethz.ch/courses/aml/
\item https://las.inf.ethz.ch/pai-f19
\item https://www.coursera.org/learn/machine-learning/lecture/zcAuT/welcome-to-machine-learning
\end{itemize}

Regression: predict real-valued output

Classification: discrete valued outputs

\section{Supervised Learning}

Training set - with $m$ number of training examples, $x$ input variables / features, $y$ outputs/targets

$(x^{(i)},y^{(i)})$ is a training example

\subsection{ Linear Regression (Uni-variable)

Hypothesis: $h_\theta(x) = \theta_0 + \theta_1x$ 

Cost function: a function that measures the performance of the hypothesis

For linear regression: 

$$\min_{\theta_0,\theta_1} \frac{1}{2m}\sum_i| h_\theta(x^{(i)})-y^{(i)} |^2 $$

Squared Error Cost Function: $J = \frac{1}{2m}\sum_i( h_\theta(x^{(i)})-y^{(i)} )^2 $ 

\subsubsection{Gradient Descent for Linear Regression}

For linear regression - the least squared cost function has no local minimum

GD will converge 

Normal Equations can be used to perform a single step solution for linear models, but GD scales better for large training sets

\subsubsection{Stochastic Gradient Descent

Computes the gradient with respect to each training example directly and aggregates it. 

Can  converge to a minimum much faster than batch gradient descent 

\subsection{Multi Variable Linear Regression

For $n$ features, define $ x\in\mathbb{R}^{n+1}$, $0th$ indexed vector,  the features vector, where $x_0^{(i)} := 1 \forall i$  

And $\theta = (0_0,\dots,\theta_n)$  the model

The hypothesis: $h_{\theta} = \sum \theta_i x_i = \theta^T x$  

Update rule for linear regression:

$\theta_0 = \theta_0 -\alpha\frac{1}{m}\sum_i (h_\theta(x^{(i)})-y^{(i)})\cdot x_0$

and similarly for all other variables

\subsubsection{Feature Scaling}

If features are of very different dimensions, the cost function will have skewed contours in the energy landscape. The gradient descent has this ping-pong behavior.  

It helps to scale the parameters to approx. $-1 \le x_j^{(i)}\le 1$

\subsubsection{Mean Normalization}

Replace $x_i$ with $x_i - \mu_i$ to make the variable approx. 0-mean

$x_i \leftarrow \frac{x_i-\mu_i}{range}$

s = Range will be $\text{max}-\text{min}$

\subsection{Debugging Gradient Descent}

Plot the cost function when GD runs

Num of iterations depends on the algorithm / model

Automatic convergence tests:

\begin{itemize}
\item change in $J(\theta)$ decreases by less than $10^{-3}$
\end{itemize}

If the cost function value increases, try smaller $\alpha$

When visualization - either wave behavior or increase in the model

Gradient verification with FD

If $\alpha$ is small enough, GD should decrease for every iteration

![convergence](/Users/kozlovy/Documents/2019_JobApplications/Notes/ml_figures/convergence.png)

A - good convergence

B - slow convergence

C - learning rate too high

Run GD with $\alpha$ with a range of values with 10-scale factor  3x from previous values

Until you find one value which is too small and one value which is too large

\subsubsection{Momentum}

\subsubsection{Netwon}

\begin{itemize}
\item no learning rate
\end{itemize}

For a function $l$ with the derivative $l^\prime(\theta)$ and second derivative, starting from an initial guess the update rule is:

$\theta := \theta - \frac{l^\prime(\theta)}{l^{\prime\prime}\theta}$

until $l^\prime(\theta)=0$. 

Newton method looks at the approximated tangent to $l(\theta)$ at the point $\theta$ and solves for where the line is equal to 0.

\subsubsection{Newton Raphson Method}

Generalization of Netwon's method to multi-variable / multi dimension settings:

$\theta = \theta - H^{-1}\nabla_\theta l(\theta)$

where $\nabla_\theta$ is a vector of partial derivatives of $l(\theta)$ with respect to $\theta$ and 

$H(\theta)=\frac{\partial^2 l(\theta)}{\partial \theta_i\partial\theta_j}$

Better and faster convergence than GD, but expensive, requires (careful) evaluation, Hessian needs to be invertible (full rank)

Fischer scoring - applying Newton's to logistic regression log likelihood function

\subsection{Polynomial Regression}

Basically, the idea here is to cheat and pre-compute the feature vector. 

For example, $(x_1 := x, x_2 := x^2, x_3 := x^3)$. 

The previous formulation and update rules hold: $\theta^Tx$

In this case it's important to scale the variables!

Other options: sqrt, cubic, squared (which might not fit a lot of models)

\subsection{Normal Equation}

For a feature vector $n$ features and $m$ data points: 

Construct a matrix $X  \in \mathbb{R}^{m\times (n+1)}$ which contains all of features for all the variables + (n+1) column which contains all 1s.

$\left( \begin{matrix} 1 & x_1^1 & ... & x_1^n \\ \vdots & x_2^1 & ... & x_2^n \\   1 & x_m^1 & ... & x_m^n \\ \end{matrix} \right)$

And collect all of the observations in a vector $y \in \mathbb{R}^m $:

And we solve for a model:

$\theta = (X^T X)^{-1} X^T y$



Now, this is true only if $X^T X$ is invertible

Feature scaling is not necc. when using the normal equation.

\subsection{GD vs. Normal Equation}

GD
\begin{itemize}
\item need to choose learning rate
\item need many iterations
\item works well when $n$ is large
\end{itemize}

Normal Equation 
\begin{itemize}
\item slow for large $n$  $O(n^3)$, n=10k is where switching over could be beneficial
\item no need to choose learning rate
\item direct
\end{itemize}

\subsection{When is $X^TX$ non-invertible?}

\begin{itemize}
\item linearly dependent features - i.e. size in m^2 and size in feet squared
\begin{itemize}
\item remove features
\end{itemize}
\item too many features $n\ge m$ 
\begin{itemize}
  \item delete features 
  \item use regularization
\end{itemize}
\end{itemize}

\section{Classification}

\subsection{Two Class Problems}

<img src="/Users/kozlovy/Documents/2019_JobApplications/Notes/ml_figures/log_threshold.png" alt="log_threshold" style="zoom:33%;" />

Using linear regression model + threshold:

Classification is not actually a linear function - using linear models doesn't work well.

Labels are usually {0,1} known as negative and positive classes.

\subsubsection{Logistic Regression}

Want a model that predict a value $0\le h_\theta(x)\le 1$

Model: $h_\theta(x)=g(\theta^T x)$ 

Logistic/sigmoid function: $g(z) = \frac{1}{1+e^{-z}}$

Together: $h_\theta(x)=\frac{1}{1+e^{-\theta^T x}}$

![sigmoid](/Users/kozlovy/Documents/2019_JobApplications/Notes/ml_figures/sigmoid.png)

Has  asymptotes at {0,1}

\subsubsection{Interpretation of Output}

$h_\theta(x)$ is the estimated probability that  $y=1$ on input $x$ 

$h_\theta(x) = P(y=1|x;\theta)$ 

$P(y=0|x;\theta) = 1-P(y=1|x;\theta)$

\subsubsection{Decision Boundary}

$g(z) \ge 0$.5 when $z>0$ 

$g(\theta^T x ) \ge 0.5$ when $\theta^Tx \ge 0$

(basically, here we  can derive this from $1+e^{-\theta^T x}  = 2$

The decision boundary is a function of the hypothesis and its parameters

\subsubsection{Non Linear Decision Boundaries}

Can perform a similar trick as with linear regression -> polynomial regression - build features such as $x_1^2$ etc...

So for example: 

$$\theta = \left[ \begin{matrix} -1 & 0 & 0  & 1 & 1 \end{matrix} \right]$$

$$h_\theta(x) = g(\theta^T(1,x_1,x_2,x_1^2,x_2^2 )) $$

The decision boundary will lie at $x_1^2 + x_2^2 = 1$

\subsubsection{Cost Function}

Using the linear regression cost function is non convex for the logistic regression.

$Cost(h_\theta(x),y) = \begin{cases} -\log(h_\theta(x))  ;\ \text{if} \;  y=1 \\ -\log(1-h_\theta(x))   ;\ \text{if} \;  y=0  \end{cases}$

This formulation has desirable properties: 

$(h(x)=0, y = 0)$ or$(h(x) = 1, y = 1)$  - cost = 0

Very high penalization if $(h(x)=1, y = 0)$ or$(h(x) = 0, y = 1)$ due to the cost function going asymptotically to $\infty$ :

<img src="/Users/kozlovy/Documents/2019_JobApplications/Notes/ml_figures/cost_function.png" alt="cost_function" style="zoom:25%;" />

\subsubsection{Simplified Cost Function}

A generalized cost function is: 

$Cost(h_\theta (x), y) =-y\log(h_\theta(x))-(1-y)\log(1-h_\theta(x))  $

And summarizing over all examples:

$J(\theta)= -\frac{1}{m} \sum_{i=1}^{m} y^{(i)}\log(h_\theta(x^{(i)}))+(1-y^{(i)})\log(1-h_\theta(x^{(i)}))$

To minimize, solve for parameters:

$\min_{\theta} J(\theta) $ 

Output / new prediction: $h_\theta(x) = \frac{1}{1+e^{-\theta^T x}}$

$\frac{\partial}{\partial_{\theta_j}} J(\theta) = \frac 1 m \sum_i (h_\theta(x^{(i)})-y^{(i)})x_j^{(i)}$

Exactly the same update as linear regression. Here the main difference is that $h_\theta$  went from $\theta^T x $ to $\frac{1}{1+e^{-\theta^Tx}}$

And the update rules are:

$\theta_j := \theta_j - \frac{\alpha}{m} \sum_{i=1}^m (h_\theta(x^{(i)}) - y^{(i)}) x_j^{(i)}$

And vectorized:

$\theta:=\theta - \frac{\alpha}{m}X^T(g(X\theta)-\vec{y})$

\subsection{ Maximum Likelihood Estimation + Convexity

Convexity: gives us lower bounds on the first order approximation of the function (i.e. the first order approximation is guaranteed to be larger than or equal to the real function value).

Assuming that the target variables and input are related via the equation: 

$y^{(i)}=\theta^Tx^{(i)}+\epsilon^{(i)}$

where $\epsilon$ are IID (independently and identically distributed) error terms the captures unmodeled effects, i.e random noise.

Assuming $e^{i}\sim\mathcal{N}(0,\sigma^2)=\frac{1}{\sqrt{2\pi}\sigma}\exp\left(-\frac{(\epsilon^{(i)})^2}{2\sigma^2}\right)$

That implies that: $p(y^{(i)}|x^{(i)};\theta)=\frac{1}{\sqrt{2\pi}\sigma}\exp\left(-\frac{(y^{i}- \theta^Tx^{(i)})^2}{2\sigma^2}\right)$ - this does not depend on $\theta$, the model is not a random variable! 

For the entire model's training set $X$  we can define this the **likelihood** function of the model : $L(\theta)=L(\theta;X;\vec{y})=p(\vec{y}|X;\theta)$

$L(\theta) = L(\theta;X,\vec y) = p(\vec y| X;\theta)$

Since all of the observations are independent:

$L(\theta)= \Pi_{i} p(y^{(i)}| x^{(i)};\theta) = \Pi_{i} \frac{1}{\sqrt{2\pi}\sigma}\exp\left(-\frac{(y^{i}- \theta^Tx^{(i)})^2}{2\sigma^2}\right) $

Maximum likelihood: we should choose a model $\theta$ so as maximize the probability of the data: $\theta$ should maximize $L(\theta)$. 

By deriving the function that maximizes $\log L(\theta)$ , product becomes a series sum and we simply need to maximize the $\frac 1 2 \sum_i (y^{(i)}-\theta^T x^{(i)})^2$ which is the original least-squares cost function.

Note that this does not depend on $\sigma$ !

\subsubsection{Maximum A Posteriori}

/TODO

\subsection{Locally Weighted Linear Models}

\subsection{Optimization Techniques}

There following algorithms are alternatives to GD that do not require choosing a learning rate:

\begin{itemize}
\item Conjugate Gradient
\item BFGS
\item L-BFGS
\end{itemize}

Advantages:
\begin{itemize}
\item No learning rate
\item Faster than GD
\item Line search
\end{itemize}

Disadvantages
\begin{itemize}
\item More complex
\item Prob. don't imp. yourself
\end{itemize}

\subsubsection{Multi-Class Classification Problems}

\subsubsection{One vs. All}

For example: tagging emails according to multiple classes; weather (rainy, sunny)

For each class, train a logistic regression classifier $h_{\theta}^{(i)}(x)$ that predicts that probability that $y=i$.

For new input choose $\max_ih_\theta^i(x)$

\section{ Overfitting vs. Bias}

![overfitting](/Users/kozlovy/Documents/2019_JobApplications/Notes/ml_figures/overfitting.png)Underfitting -> high bias. 

Overfitting, high variance

High variance - fitting a high order polynomial can be used to fit almost any function, not enough data to give a good hypothesis

\item If we have too many features, the learned hypothesis may fit the training data very well, but fail to generalize

\subsubsection{Addressing Overfitting}

Reduce number of features

\item requires deciding which feature to keep and discard
\item model selection algorithms

Regularization

\item keep features but reduce magnitude / values of $\theta_j$ 
\item works well when there are a lot features, each of which contributes less

Modify the cost function by penalizing the parameters:

Penalize higher order parameters: equiv to reducing the model to lower order model - simplfying the model

Penalize all parameters  - trying to keep the hypothesis small, usually corresponds to smoother functions

So now the objective has a data term and a regularization term.

The regularization term: $\lambda\sum_{j=1} \theta_j^2 $ keeps all of them small

If $\lambda $ is very large, in linear reg., all model params will be close to 0 and $h_\theta(x) = \theta_0$  


\section{Measuring Model Performance}

Type 1 Error - False positive - Predict an event when there was no event
Type 2 Error - False negative - Predict no event when in fact there was an event.

\subsubsection{Precision-Recall}

Precision-Recall curves summarize the trade-off between the true positive rate and the positive predictive value for a predictive model using different probability thresholds.

Precision-recall curves are appropriate for imbalanced datasets.

\subsubsection{ ROC -  Receiver Operating Characteristic curve

Summarize the trade-off between the true positive rate and false positive rate for a predictive model using different probability thresholds.

ROC curves are appropriate when the observations are balanced between each class

**Convolution** is a [mathematical operation](https://en.wikipedia.org/wiki/Operation_(mathematics)) on two [functions](https://en.wikipedia.org/wiki/Function_(mathematics)) (*f* and *g*) that produces a third function expressing how the shape of one is modified by the other

$(f\star g)(t) = \int_{-\infty}^{\infty} f(\tau)\cdot g(t-\tau)d\tau = \int_{-\infty}^{\infty} f(t-\tau)\cdot g(\tau)d\tau $

Commutative. 

For functions which only have limited support the integration is only done on the valid domain.

\subsection{L1 vs L2 Norm}

L2 norm strongly penalizes outliers. For good data with some very far outlier it might not generate the "best" fit as judged by a human observer.

L1 favors sparse coefficients.



\section{Support Vector Machines}

https://medium.com/machine-learning-101/chapter-2-svm-support-vector-machine-theory-f0812effc72\section{targetText=A Support Vector Machine (SVM,hyperplane which categorizes new examples.))



\section{Decision Trees}
Recursive repartition of the data

\subsection{Random Forest Regression}

An ensemble of decision trees. During learning tree nodes are split using random variable subset of data features.

All trees vote to produce final result.

For best results trees should be as independent as possible. Splitting using a random subset of features achieves this.

Averaging the product of the trees reduces overfitting to noise

5-100 Trees.

\subsection{Random Fern Regressors}



\section{Boosting}
Learning strong classifiers from weak classifiers.





\section{Naive Bayes Classifier}

/TODO

\section{RANSAC}

A method for dealing with noisy data. 

Partition the method 

Is not determinant, depends on the subset selection, and is not guaranteed to converge.

1. Select a random subset of the original data. Call this subset the *hypothetical inliers*.
2. A model is fitted to the set of hypothetical inliers.
3. All other data are then tested against the fitted model. Those points that fit the estimated model well, according to some model-specific [loss function](https://en.wikipedia.org/wiki/Loss_function), are considered as part of the *consensus set*.
4. The estimated model is reasonably good if sufficiently many points have been classified as part of the consensus set.
5. Afterwards, the model may be improved by reestimating it using all members of the consensus set.

```
Given:
    data – a set of observations
    model – a model to explain observed data points
    n – minimum number of data points required to estimate model parameters
    k – maximum number of iterations allowed in the algorithm
    t – threshold value to determine data points that are fit well by model 
    d – number of close data points required to assert that a model fits well to data

Return:
    bestFit – model parameters which best fit the data (or nul if no good model is found)

iterations = 0
bestFit = nul
bestErr = something really large
while iterations < k {
    maybeInliers = n randomly selected values from data
    maybeModel = model parameters fitted to maybeInliers
    alsoInliers = empty set
    for every point in data not in maybeInliers {
        if point fits maybeModel with an error smaller than t
             add point to alsoInliers
    }
    if the number of elements in alsoInliers is > d {
        % this implies that we may have found a good model
        % now test how good it is
        betterModel = model parameters fitted to all points in maybeInliers and alsoInliers
        thisErr = a measure of how well betterModel fits these points
        if thisErr < bestErr {
            bestFit = betterModel
            bestErr = thisErr
        }
    }
    increment iterations
}
return bestFit
```



\section{Bagging/Boosting}

Collaborative filtering

\section{Generative Models}

\section{Dimension Reduction}

\subsection{PCA}

\chapter{Unsupervised Learning}

Algorithms for finding structure in data.

\section{Clustering}

The clustering problem: given an unlabeled data set, group the data into coherent  subsets or into coherent clusters for us.

\subsection{K Means}

\item $K$ number of clusters + initialization
\item Training set ${x^{(1)},x^{(2)}\dots,x^{(m)}}$
\item $x\in\mathbb{R}^n$
\item By convention, drop $x_0=1$

```
Randomly initialize K cluster centers
While not converged:
1. iterate over data and assign a cluster for each data point based on distance to center
2. re-compute the cluster mean
```

If a cluster becomes empty - remove the cluster

Or randomly re-initialize the cluster

\subsubsection{K Means for Non Separated Clusters}

\subsubsection{K Means Cost Function}

Assuming: 

$c^{(i)}$ index of cluster to which the example $x^{(i)}$ belongs to.

$\mu_k \in \mathbb R ^n $ cluster centroid 

$\mu_{c^{(i)}} \in \mathbb R ^n $ location of the cluster centroid to which example $x^{(i)}$ has been assigned

Example cost for point $Cost(x^{(i)}) = \|x^{(i)} - \mu_{c^{(i)}} \|^2$ 

$J(c^{(1)},\dots, c^{(k)})= \frac{1}{m}\sum_i \|x^{(i)} - \mu_{c^{(i)}} \|^2 $

The objective is to minimize the cost function *distortion* with respect to the clusters (both labelling and centers).

So what k-means algorithm is actually doing is:

1. minimize the cost function with respect to cluster assignments $c^{(i)}$
2. minimize the cost function with respect to cluster centroids $\mu_k$ 

(so basically block coordinate descent?)

\subsubsection{Random Initialization}

\begin{itemize}
\item $K < m$
\item Randomly pick $K$ training examples and set the cluster means to these examples
\end{itemize}

K-mean can get stuck in a local optima - to avoid this a good option is to run k-mean multiple times and get as good global optimum

For multiple initializations - run K-means loads of times, pick the clustering which results in the lowest cost function

This works well for small $K < 10$ .

For large $K$s it is not as effective.

\subsubsection{Number of Clusters - Elbow Method}

Choosing the right K 

Plot the cost function with respect to the number of clusters.![elbow](/Users/kozlovy/Documents/2019_JobApplications/Notes/ml_figures/elbow.png)

In practice it is usually a bit harder, and it is not clear that there is such a transition where the distortion stops.

\section{Dimensionality Reduction}

\subsection{PCA}

PCA is trying to find a lower dimension representation of that data which minimizes the squared distance error of the data from the representation.

Before PCA it is standard practice to perform mean normalization and feature scaling. 

\subsubsection{PCA vs Linear Regression}

<img src="/Users/kozlovy/Documents/2019_JobApplications/Notes/ml_figures/PCA_Linear.png" alt="PCA_Linear" style="zoom:50%;" />

We do not treat $y$ as a special variable

Minimized projected error vs. minimize distance from line

\section{GMM and EM}


\section{Data Generation Using Simulation}

Generating good synthetic data: 
realism, 
diverse,


Want to render images which are as different as possible from each other

Parametric model of humans - procedural generation

\section{Neural Networks}

http://karpathy.github.io/neuralnets/


\section{ML Algorithm Design}

General process of building a ML product:

\begin{enumerate}
\item What is the objective? prediction, recommendation, clustering, search, etc.
\item Pick the right algorithm: supervised vs unsupervised, classification vs regression, generalized linear model / decision tree / neural network / etc.
\item Pick / engineer relevant features based on available data.
\item Pick metrics for model performance.
\item Optionally, comment on how to optimize the model for production.
\end{enumerate}
